\documentclass[15pt, mathserif]{beamer}

\usepackage[french]{babel}
\usepackage[T1]{fontenc}
\usepackage[utf8]{inputenc}
%\usepackage{esvect}
\usepackage{bm}
\usepackage{eurosym}
\usepackage{tikz}
\usepackage{pgf,tikz,pgfplots}
\pgfplotsset{compat=1.15}
\usepackage{mathrsfs}
\usetikzlibrary{arrows}
\usetikzlibrary{arrows.meta}

\usetikzlibrary{mindmap}
\usepackage{multicol}
\usepackage[tikz]{bclogo}
\usepackage{tkz-tab}
\usepackage{amsmath, tabu}
\usepackage{esvect} %\vv{AB} pour le vecteur AB
%% Tableau

\usepackage{makecell}
\setcellgapes{1pt}
\makegapedcells
\newcolumntype{R}[1]{>{\raggedleft\arraybackslash }b{#1}}
\newcolumntype{L}[1]{>{\raggedright\arraybackslash }b{#1}}
\newcolumntype{C}[1]{>{\centering\arraybackslash }b{#1}}


%pour avoir des parenthèses rondes dans le package fourier
\DeclareSymbolFont{cmoperators}   {OT1}{cmr} {m}{n}
\DeclareSymbolFont{cmlargesymbols}{OMX}{cmex}{m}{n}

\usefonttheme{professionalfonts} %permet d'enlever un bug avec fourier
\usepackage{fourier}
\DeclareMathDelimiter{(}{\mathopen} {cmoperators}{"28}{cmlargesymbols}{"00}
\DeclareMathDelimiter{)}{\mathclose}{cmoperators}{"29}{cmlargesymbols}{"01}

%Graphiques 

\usepackage{pgf,tikz,pgfplots}
\pgfplotsset{compat=1.15}
\usepackage{mathrsfs}
\usetikzlibrary{arrows}
\usetikzlibrary{mindmap}

%ensembles de nbres

\newcommand{\R}{\mathbb{R}}			%permet d'écrire le R "ensemble des réels"'
\newcommand{\N}{\mathbb{N}}			%permet d'écrire le N "ensemble des entiers naturels"
\newcommand{\Z}{\mathbb{Z}}			%permet d'écrire le Z "ensemble des entiers relatifs"
\newcommand{\Prem}{\mathbb{P}}	%permet d'écrire le P "ensemble des nombres premiers" (qui n'a pas marché avec le \P car il existe déjà)
\newcommand{\D}{\mathbb{D}}
\newcommand{\Df}{\mathcal{D}_f}
\newcommand{\Cf}{\mathcal{C}_f}

\newcommand{\Q}{\mathbb{Q}}

\usetheme{Madrid}
\useoutertheme{miniframes} % Alternatively: miniframes, infolines, split
\useinnertheme{circles}
\definecolor{UBCblue}{rgb}{0.1, 0.25, 0.4} % UBC Blue (primary)
\definecolor{bordeaux}{RGB}{128,0,0}
\usecolortheme[named=UBCblue]{structure}

\usepackage{color} % J'aime bien définir mes couleurs
\definecolor{propcolor}{rgb}{0, 0.5, 1}
\colorlet{louis}{blue!70!green!60!white}
\colorlet{sakura}{pink!40!red}

\title{Activités Mentales}
\date{03 Septembre 2022}

\newcommand{\vco}[2]{\begin{pmatrix} #1 \\ #2 \end{pmatrix}} %Coordonnées de vecteur
\newenvironment{eq}{\begin{cases}\begin{tabu}{ccccc}}{\end{tabu}\end{cases}}
\newenvironment{eql}{\begin{cases}\begin{tabu}{cccccl}}{\end{tabu}\end{cases}}
\newenvironment{eqrl}{\begin{cases}\begin{tabu}{rl}}{\end{tabu}\end{cases}}


\begin{document}

\begin{frame}
    \titlepage
\end{frame}

\begin{frame} 
	\frametitle{Question 1}
Déterminer la mesure principale de l'angle $\dfrac{75\pi}{4}$ puis le placer sur le cercle trigonométrique.\end{frame}


\begin{frame} 
	\frametitle{Question 2}
Déterminer la mesure principale de l'angle $\dfrac{-102\pi}{5}$ puis le placer sur le cercle trigonométrique.\end{frame}


\begin{frame} 
	\frametitle{Question 3}
Déterminer la mesure principale de l'angle $\dfrac{-47\pi}{7}$ puis le placer sur le cercle trigonométrique.\end{frame}


\begin{frame} 
	\frametitle{Question 4}
Déterminer la mesure principale de l'angle $\dfrac{104\pi}{9}$ puis le placer sur le cercle trigonométrique.\end{frame}


\begin{frame} 
	\frametitle{Question 5}
Déterminer la mesure principale de l'angle $\dfrac{-55\pi}{7}$ puis le placer sur le cercle trigonométrique.\end{frame}


\begin{frame} 
	\frametitle{Question 6}
Déterminer la mesure principale de l'angle $\dfrac{87\pi}{4}$ puis le placer sur le cercle trigonométrique.\end{frame}


\begin{frame} 
	\frametitle{Question 7}
Déterminer la mesure principale de l'angle $\dfrac{-58\pi}{3}$ puis le placer sur le cercle trigonométrique.\end{frame}


\begin{frame} 
	\frametitle{Question 8}
Déterminer la mesure principale de l'angle $\dfrac{49\pi}{5}$ puis le placer sur le cercle trigonométrique.\end{frame}


\begin{frame} 
	\frametitle{Question 9}
Déterminer la mesure principale de l'angle $\dfrac{104\pi}{5}$ puis le placer sur le cercle trigonométrique.\end{frame}


\begin{frame} 
	\frametitle{Question 10}
Déterminer la mesure principale de l'angle $\dfrac{-12\pi}{5}$ puis le placer sur le cercle trigonométrique.\end{frame}


\begin{frame} 
	\frametitle{Question 11}
Déterminer la mesure principale de l'angle $\dfrac{131\pi}{4}$ puis le placer sur le cercle trigonométrique.\end{frame}


\begin{frame} 
	\frametitle{Question 12}
Déterminer la mesure principale de l'angle $\dfrac{-119\pi}{11}$ puis le placer sur le cercle trigonométrique.\end{frame}


\begin{frame} 
	\frametitle{Question 13}
Déterminer la mesure principale de l'angle $\dfrac{-105\pi}{11}$ puis le placer sur le cercle trigonométrique.\end{frame}


\begin{frame} 
	\frametitle{Question 14}
Déterminer la mesure principale de l'angle $\dfrac{-118\pi}{7}$ puis le placer sur le cercle trigonométrique.\end{frame}


\begin{frame}
\vspace{-10mm}
	\frametitle{Correction 1}
\begin{minipage}{0.45 \linewidth}
	\begin{align*}
		\dfrac{75\pi}{4} &= \dfrac{75}{4}\times \dfrac{2\pi}{2} \\
		&=\dfrac{75}{8} \times 2 \pi\\
		&=\dfrac{(8\times 9+3) \times 2 \pi}{8}\\
		&=\dfrac{8\times 9 \times 2 \pi}{8}+\dfrac{3\times 2\pi}{8}\\
		&=9\times 2\pi+\dfrac{3\pi}{4}
	\end{align*}
\end{minipage}
\hfil
\begin{minipage}{0.5 \linewidth}
	\begin{tikzpicture}[scale = 0.65]
		\draw[thick] (0,0) circle (2);
		\draw[-{Straight Barb[length = 0.5mm]}] (-2.25,0) -- (2.25, 0);
		\draw[-{Straight Barb[length = 0.5mm]}] (0,-2.25) -- (0, 2.25);
		\begin{scope}[rotate = 90.0]
	\draw[dotted] (0,0) -- (2,0);
	\draw[thick] (1.9, 0) -- (2.1,0);
	\end{scope}

\begin{scope}[rotate = 225.0]
	\draw[dotted] (0,0) -- (2,0);
	\draw[thick] (1.9, 0) -- (2.1,0);
	\end{scope}

\begin{scope}[rotate = 270.0]
	\draw[dotted] (0,0) -- (2,0);
	\draw[thick] (1.9, 0) -- (2.1,0);
	\end{scope}

\begin{scope}[rotate = 315.0]
	\draw[dotted] (0,0) -- (2,0);
	\draw[thick] (1.9, 0) -- (2.1,0);
	\end{scope}

\begin{scope}[rotate = 135.0]
	\draw[dotted, louis, thick] (0,0) -- (2,0);
	\draw[louis, thick] (1.9, 0) -- (2.1,0);
	\draw[louis] (2.3, 0) node [above, left] {$\dfrac{75\pi}{4}=\dfrac{3\pi}{4}$};
\end{scope}

\draw[dashed, louis, thick] ({2*cos(deg 3/4*pi)},0) -- ({2*cos(deg 3/4*pi)}, {2*sin(deg 3/4*pi)}) -- (0, {2*sin(deg 3/4*pi)});\begin{scope}[rotate = 45.0]
	\draw[thick, dotted, louis] (0,0) -- (2,0);
	\draw[thick, louis] (1.9, 0) -- (2.1,0) node[above right] {$\dfrac{\pi}{4}$};
\end{scope}

\draw[dashed, louis, thick] ({2*cos(deg 1/4*pi)},0) -- ({2*cos(deg 1/4*pi)}, {2*sin(deg 1/4*pi)}) -- (0, {2*sin(deg 1/4*pi)});\end{tikzpicture}
\end{minipage}

Comme $-\pi < \dfrac{3\pi}{4}\leq \pi$, la mesure principale de $\dfrac{75\pi}{4}$ est $\dfrac{-3\pi}{4}$.\end{frame}


\begin{frame}
\vspace{-10mm}
	\frametitle{Correction 2}
\begin{minipage}{0.45 \linewidth}
	\begin{align*}
		\dfrac{-102\pi}{5} &= \dfrac{-102}{5}\times \dfrac{2\pi}{2} \\
		&=\dfrac{-102}{10} \times 2 \pi\\
		&=\dfrac{-(10\times 10+2) \times 2 \pi}{10}\\
		&=-\dfrac{10\times 10 \times 2 \pi}{10}-\dfrac{2\times 2\pi}{10}\\
		&=-10\times 2\pi-\dfrac{2\pi}{5}
	\end{align*}
\end{minipage}
\hfil
\begin{minipage}{0.5 \linewidth}
	\begin{tikzpicture}[scale = 0.65]
		\draw[thick] (0,0) circle (2);
		\draw[-{Straight Barb[length = 0.5mm]}] (-2.25,0) -- (2.25, 0);
		\draw[-{Straight Barb[length = 0.5mm]}] (0,-2.25) -- (0, 2.25);
		\begin{scope}[rotate = 72.0]
	\draw[dotted] (0,0) -- (2,0);
	\draw[thick] (1.9, 0) -- (2.1,0);
	\end{scope}

\begin{scope}[rotate = 108.0]
	\draw[dotted] (0,0) -- (2,0);
	\draw[thick] (1.9, 0) -- (2.1,0);
	\end{scope}

\begin{scope}[rotate = 144.0]
	\draw[dotted] (0,0) -- (2,0);
	\draw[thick] (1.9, 0) -- (2.1,0);
	\end{scope}

\begin{scope}[rotate = 216.0]
	\draw[dotted] (0,0) -- (2,0);
	\draw[thick] (1.9, 0) -- (2.1,0);
	\end{scope}

\begin{scope}[rotate = 252.0]
	\draw[dotted] (0,0) -- (2,0);
	\draw[thick] (1.9, 0) -- (2.1,0);
	\end{scope}

\begin{scope}[rotate = 324.0]
	\draw[dotted] (0,0) -- (2,0);
	\draw[thick] (1.9, 0) -- (2.1,0);
	\end{scope}

\begin{scope}[rotate = -72.0]
	\draw[dotted, louis, thick] (0,0) -- (2,0);
	\draw[louis, thick] (1.9, 0) -- (2.1,0);
	\draw[louis] (2.3, 0) node [below, right] {$\dfrac{-102\pi}{5}=\dfrac{-2\pi}{5}$};
\end{scope}

\draw[dashed, louis, thick] ({2*cos(deg -2/5*pi)},0) -- ({2*cos(deg -2/5*pi)}, {2*sin(deg -2/5*pi)}) -- (0, {2*sin(deg -2/5*pi)});\begin{scope}[rotate = 36.0]
	\draw[thick, dotted, louis] (0,0) -- (2,0);
	\draw[thick, louis] (1.9, 0) -- (2.1,0) node[above right] {$\dfrac{\pi}{5}$};
\end{scope}

\draw[dashed, louis, thick] ({2*cos(deg 1/5*pi)},0) -- ({2*cos(deg 1/5*pi)}, {2*sin(deg 1/5*pi)}) -- (0, {2*sin(deg 1/5*pi)});\end{tikzpicture}
\end{minipage}

Comme $-\pi < -\dfrac{2\pi}{5}\leq \pi$, la mesure principale de $\dfrac{-102\pi}{5}$ est $\dfrac{-2\pi}{5}$.\end{frame}


\begin{frame}
\vspace{-10mm}
	\frametitle{Correction 3}
\begin{minipage}{0.45 \linewidth}
	\begin{align*}
		\dfrac{-47\pi}{7} &= \dfrac{-47}{7}\times \dfrac{2\pi}{2} \\
		&=\dfrac{-47}{14} \times 2 \pi\\
		&=\dfrac{-(14\times 3+5) \times 2 \pi}{14}\\
		&=-\dfrac{14\times 3 \times 2 \pi}{14}-\dfrac{5\times 2\pi}{14}\\
		&=-3\times 2\pi-\dfrac{5\pi}{7}
	\end{align*}
\end{minipage}
\hfil
\begin{minipage}{0.5 \linewidth}
	\begin{tikzpicture}[scale = 0.65]
		\draw[thick] (0,0) circle (2);
		\draw[-{Straight Barb[length = 0.5mm]}] (-2.25,0) -- (2.25, 0);
		\draw[-{Straight Barb[length = 0.5mm]}] (0,-2.25) -- (0, 2.25);
		\begin{scope}[rotate = 51.42]
	\draw[dotted] (0,0) -- (2,0);
	\draw[thick] (1.9, 0) -- (2.1,0);
	\end{scope}

\begin{scope}[rotate = 77.13]
	\draw[dotted] (0,0) -- (2,0);
	\draw[thick] (1.9, 0) -- (2.1,0);
	\end{scope}

\begin{scope}[rotate = 102.84]
	\draw[dotted] (0,0) -- (2,0);
	\draw[thick] (1.9, 0) -- (2.1,0);
	\end{scope}

\begin{scope}[rotate = 128.55]
	\draw[dotted] (0,0) -- (2,0);
	\draw[thick] (1.9, 0) -- (2.1,0);
	\end{scope}

\begin{scope}[rotate = 154.26]
	\draw[dotted] (0,0) -- (2,0);
	\draw[thick] (1.9, 0) -- (2.1,0);
	\end{scope}

\begin{scope}[rotate = 179.97]
	\draw[dotted] (0,0) -- (2,0);
	\draw[thick] (1.9, 0) -- (2.1,0);
	\end{scope}

\begin{scope}[rotate = 205.68]
	\draw[dotted] (0,0) -- (2,0);
	\draw[thick] (1.9, 0) -- (2.1,0);
	\end{scope}

\begin{scope}[rotate = 257.1]
	\draw[dotted] (0,0) -- (2,0);
	\draw[thick] (1.9, 0) -- (2.1,0);
	\end{scope}

\begin{scope}[rotate = 282.81]
	\draw[dotted] (0,0) -- (2,0);
	\draw[thick] (1.9, 0) -- (2.1,0);
	\end{scope}

\begin{scope}[rotate = 308.52]
	\draw[dotted] (0,0) -- (2,0);
	\draw[thick] (1.9, 0) -- (2.1,0);
	\end{scope}

\begin{scope}[rotate = 334.23]
	\draw[dotted] (0,0) -- (2,0);
	\draw[thick] (1.9, 0) -- (2.1,0);
	\end{scope}

\begin{scope}[rotate = -128.55]
	\draw[dotted, louis, thick] (0,0) -- (2,0);
	\draw[louis, thick] (1.9, 0) -- (2.1,0);
	\draw[louis] (2.3, 0) node [below, left] {$\dfrac{-47\pi}{7}=\dfrac{-5\pi}{7}$};
\end{scope}

\draw[dashed, louis, thick] ({2*cos(deg -5/7*pi)},0) -- ({2*cos(deg -5/7*pi)}, {2*sin(deg -5/7*pi)}) -- (0, {2*sin(deg -5/7*pi)});\begin{scope}[rotate = 25.71]
	\draw[thick, dotted, louis] (0,0) -- (2,0);
	\draw[thick, louis] (1.9, 0) -- (2.1,0) node[above right] {$\dfrac{\pi}{7}$};
\end{scope}

\draw[dashed, louis, thick] ({2*cos(deg 1/7*pi)},0) -- ({2*cos(deg 1/7*pi)}, {2*sin(deg 1/7*pi)}) -- (0, {2*sin(deg 1/7*pi)});\end{tikzpicture}
\end{minipage}

Comme $-\pi < -\dfrac{5\pi}{7}\leq \pi$, la mesure principale de $\dfrac{-47\pi}{7}$ est $\dfrac{-5\pi}{7}$.\end{frame}


\begin{frame}
\vspace{-10mm}
	\frametitle{Correction 4}
\begin{minipage}{0.45 \linewidth}
	\begin{align*}
		\dfrac{104\pi}{9} &= \dfrac{104}{9}\times \dfrac{2\pi}{2} \\
		&=\dfrac{104}{18} \times 2 \pi\\
		&=\dfrac{(18\times 5+14) \times 2 \pi}{18}\\
		&=\dfrac{18\times 5 \times 2 \pi}{18}+\dfrac{14\times 2\pi}{18}\\
		&=5\times 2\pi+\dfrac{14\pi}{9}
	\end{align*}
\end{minipage}
\hfil
\begin{minipage}{0.5 \linewidth}
	\begin{tikzpicture}[scale = 0.65]
		\draw[thick] (0,0) circle (2);
		\draw[-{Straight Barb[length = 0.5mm]}] (-2.25,0) -- (2.25, 0);
		\draw[-{Straight Barb[length = 0.5mm]}] (0,-2.25) -- (0, 2.25);
		\begin{scope}[rotate = 40.0]
	\draw[dotted] (0,0) -- (2,0);
	\draw[thick] (1.9, 0) -- (2.1,0);
	\end{scope}

\begin{scope}[rotate = 60.0]
	\draw[dotted] (0,0) -- (2,0);
	\draw[thick] (1.9, 0) -- (2.1,0);
	\end{scope}

\begin{scope}[rotate = 80.0]
	\draw[dotted] (0,0) -- (2,0);
	\draw[thick] (1.9, 0) -- (2.1,0);
	\end{scope}

\begin{scope}[rotate = 100.0]
	\draw[dotted] (0,0) -- (2,0);
	\draw[thick] (1.9, 0) -- (2.1,0);
	\end{scope}

\begin{scope}[rotate = 120.0]
	\draw[dotted] (0,0) -- (2,0);
	\draw[thick] (1.9, 0) -- (2.1,0);
	\end{scope}

\begin{scope}[rotate = 140.0]
	\draw[dotted] (0,0) -- (2,0);
	\draw[thick] (1.9, 0) -- (2.1,0);
	\end{scope}

\begin{scope}[rotate = 160.0]
	\draw[dotted] (0,0) -- (2,0);
	\draw[thick] (1.9, 0) -- (2.1,0);
	\end{scope}

\begin{scope}[rotate = 200.0]
	\draw[dotted] (0,0) -- (2,0);
	\draw[thick] (1.9, 0) -- (2.1,0);
	\end{scope}

\begin{scope}[rotate = 220.0]
	\draw[dotted] (0,0) -- (2,0);
	\draw[thick] (1.9, 0) -- (2.1,0);
	\end{scope}

\begin{scope}[rotate = 240.0]
	\draw[dotted] (0,0) -- (2,0);
	\draw[thick] (1.9, 0) -- (2.1,0);
	\end{scope}

\begin{scope}[rotate = 260.0]
	\draw[dotted] (0,0) -- (2,0);
	\draw[thick] (1.9, 0) -- (2.1,0);
	\end{scope}

\begin{scope}[rotate = 300.0]
	\draw[dotted] (0,0) -- (2,0);
	\draw[thick] (1.9, 0) -- (2.1,0);
	\end{scope}

\begin{scope}[rotate = 320.0]
	\draw[dotted] (0,0) -- (2,0);
	\draw[thick] (1.9, 0) -- (2.1,0);
	\end{scope}

\begin{scope}[rotate = 340.0]
	\draw[dotted] (0,0) -- (2,0);
	\draw[thick] (1.9, 0) -- (2.1,0);
	\end{scope}

\begin{scope}[rotate = -80.0]
	\draw[dotted, louis, thick] (0,0) -- (2,0);
	\draw[louis, thick] (1.9, 0) -- (2.1,0);
	\draw[louis] (2.3, 0) node [below, right] {$\dfrac{104\pi}{9}=\dfrac{-4\pi}{9}$};
\end{scope}

\draw[dashed, louis, thick] ({2*cos(deg -4/9*pi)},0) -- ({2*cos(deg -4/9*pi)}, {2*sin(deg -4/9*pi)}) -- (0, {2*sin(deg -4/9*pi)});\begin{scope}[rotate = 20.0]
	\draw[thick, dotted, louis] (0,0) -- (2,0);
	\draw[thick, louis] (1.9, 0) -- (2.1,0) node[above right] {$\dfrac{\pi}{9}$};
\end{scope}

\draw[dashed, louis, thick] ({2*cos(deg 1/9*pi)},0) -- ({2*cos(deg 1/9*pi)}, {2*sin(deg 1/9*pi)}) -- (0, {2*sin(deg 1/9*pi)});\end{tikzpicture}
\end{minipage}

Or $\dfrac{14\pi}{9}>\pi$, on fait un tour de moins en retirant $2\pi$: $\dfrac{14\pi}{9}-2\pi = \dfrac{-4\pi}{9}$.

Comme $-\pi <-\dfrac{4\pi}{9}\leq \pi$, la mesure principale de $\dfrac{104\pi}{9}$ est $\dfrac{-4\pi}{9}$.\end{frame}


\begin{frame}
\vspace{-10mm}
	\frametitle{Correction 5}
\begin{minipage}{0.45 \linewidth}
	\begin{align*}
		\dfrac{-55\pi}{7} &= \dfrac{-55}{7}\times \dfrac{2\pi}{2} \\
		&=\dfrac{-55}{14} \times 2 \pi\\
		&=\dfrac{-(14\times 3+13) \times 2 \pi}{14}\\
		&=-\dfrac{14\times 3 \times 2 \pi}{14}-\dfrac{13\times 2\pi}{14}\\
		&=-3\times 2\pi-\dfrac{13\pi}{7}
	\end{align*}
\end{minipage}
\hfil
\begin{minipage}{0.5 \linewidth}
	\begin{tikzpicture}[scale = 0.65]
		\draw[thick] (0,0) circle (2);
		\draw[-{Straight Barb[length = 0.5mm]}] (-2.25,0) -- (2.25, 0);
		\draw[-{Straight Barb[length = 0.5mm]}] (0,-2.25) -- (0, 2.25);
		\begin{scope}[rotate = 51.42]
	\draw[dotted] (0,0) -- (2,0);
	\draw[thick] (1.9, 0) -- (2.1,0);
	\end{scope}

\begin{scope}[rotate = 77.13]
	\draw[dotted] (0,0) -- (2,0);
	\draw[thick] (1.9, 0) -- (2.1,0);
	\end{scope}

\begin{scope}[rotate = 102.84]
	\draw[dotted] (0,0) -- (2,0);
	\draw[thick] (1.9, 0) -- (2.1,0);
	\end{scope}

\begin{scope}[rotate = 128.55]
	\draw[dotted] (0,0) -- (2,0);
	\draw[thick] (1.9, 0) -- (2.1,0);
	\end{scope}

\begin{scope}[rotate = 154.26]
	\draw[dotted] (0,0) -- (2,0);
	\draw[thick] (1.9, 0) -- (2.1,0);
	\end{scope}

\begin{scope}[rotate = 179.97]
	\draw[dotted] (0,0) -- (2,0);
	\draw[thick] (1.9, 0) -- (2.1,0);
	\end{scope}

\begin{scope}[rotate = 205.68]
	\draw[dotted] (0,0) -- (2,0);
	\draw[thick] (1.9, 0) -- (2.1,0);
	\end{scope}

\begin{scope}[rotate = 231.39000000000001]
	\draw[dotted] (0,0) -- (2,0);
	\draw[thick] (1.9, 0) -- (2.1,0);
	\end{scope}

\begin{scope}[rotate = 257.1]
	\draw[dotted] (0,0) -- (2,0);
	\draw[thick] (1.9, 0) -- (2.1,0);
	\end{scope}

\begin{scope}[rotate = 282.81]
	\draw[dotted] (0,0) -- (2,0);
	\draw[thick] (1.9, 0) -- (2.1,0);
	\end{scope}

\begin{scope}[rotate = 308.52]
	\draw[dotted] (0,0) -- (2,0);
	\draw[thick] (1.9, 0) -- (2.1,0);
	\end{scope}

\begin{scope}[rotate = 334.23]
	\draw[dotted] (0,0) -- (2,0);
	\draw[thick] (1.9, 0) -- (2.1,0);
	\end{scope}

\begin{scope}[rotate = 25.71]
	\draw[dotted, louis, thick] (0,0) -- (2,0);
	\draw[louis, thick] (1.9, 0) -- (2.1,0);
	\draw[louis] (2.3, 0) node [above, right] {$\dfrac{-55\pi}{7}=\dfrac{\pi}{7}$};
\end{scope}

\draw[dashed, louis, thick] ({2*cos(deg 1/7*pi)},0) -- ({2*cos(deg 1/7*pi)}, {2*sin(deg 1/7*pi)}) -- (0, {2*sin(deg 1/7*pi)});\end{tikzpicture}
\end{minipage}

Or $-\dfrac{13\pi}{7}\leq-\pi$, on fait un tour de plus en rajoutant $2\pi$: $-\dfrac{13\pi}{7}+2\pi = \dfrac{\pi}{7}$.

Comme $-\pi <\dfrac{\pi}{7}\leq \pi$, la mesure principale de $\dfrac{-55\pi}{7}$ est $\dfrac{\pi}{7}$.\end{frame}


\begin{frame}
\vspace{-10mm}
	\frametitle{Correction 6}
\begin{minipage}{0.45 \linewidth}
	\begin{align*}
		\dfrac{87\pi}{4} &= \dfrac{87}{4}\times \dfrac{2\pi}{2} \\
		&=\dfrac{87}{8} \times 2 \pi\\
		&=\dfrac{(8\times 10+7) \times 2 \pi}{8}\\
		&=\dfrac{8\times 10 \times 2 \pi}{8}+\dfrac{7\times 2\pi}{8}\\
		&=10\times 2\pi+\dfrac{7\pi}{4}
	\end{align*}
\end{minipage}
\hfil
\begin{minipage}{0.5 \linewidth}
	\begin{tikzpicture}[scale = 0.65]
		\draw[thick] (0,0) circle (2);
		\draw[-{Straight Barb[length = 0.5mm]}] (-2.25,0) -- (2.25, 0);
		\draw[-{Straight Barb[length = 0.5mm]}] (0,-2.25) -- (0, 2.25);
		\begin{scope}[rotate = 90.0]
	\draw[dotted] (0,0) -- (2,0);
	\draw[thick] (1.9, 0) -- (2.1,0);
	\end{scope}

\begin{scope}[rotate = 135.0]
	\draw[dotted] (0,0) -- (2,0);
	\draw[thick] (1.9, 0) -- (2.1,0);
	\end{scope}

\begin{scope}[rotate = 225.0]
	\draw[dotted] (0,0) -- (2,0);
	\draw[thick] (1.9, 0) -- (2.1,0);
	\end{scope}

\begin{scope}[rotate = 270.0]
	\draw[dotted] (0,0) -- (2,0);
	\draw[thick] (1.9, 0) -- (2.1,0);
	\end{scope}

\begin{scope}[rotate = -45.0]
	\draw[dotted, louis, thick] (0,0) -- (2,0);
	\draw[louis, thick] (1.9, 0) -- (2.1,0);
	\draw[louis] (2.3, 0) node [below, right] {$\dfrac{87\pi}{4}=\dfrac{-\pi}{4}$};
\end{scope}

\draw[dashed, louis, thick] ({2*cos(deg -1/4*pi)},0) -- ({2*cos(deg -1/4*pi)}, {2*sin(deg -1/4*pi)}) -- (0, {2*sin(deg -1/4*pi)});\begin{scope}[rotate = 45.0]
	\draw[thick, dotted, louis] (0,0) -- (2,0);
	\draw[thick, louis] (1.9, 0) -- (2.1,0) node[above right] {$\dfrac{\pi}{4}$};
\end{scope}

\draw[dashed, louis, thick] ({2*cos(deg 1/4*pi)},0) -- ({2*cos(deg 1/4*pi)}, {2*sin(deg 1/4*pi)}) -- (0, {2*sin(deg 1/4*pi)});\end{tikzpicture}
\end{minipage}

Or $\dfrac{7\pi}{4}>\pi$, on fait un tour de moins en retirant $2\pi$: $\dfrac{7\pi}{4}-2\pi = \dfrac{-\pi}{4}$.

Comme $-\pi <-\dfrac{\pi}{4}\leq \pi$, la mesure principale de $\dfrac{87\pi}{4}$ est $\dfrac{-\pi}{4}$.\end{frame}


\begin{frame}
\vspace{-10mm}
	\frametitle{Correction 7}
\begin{minipage}{0.45 \linewidth}
	\begin{align*}
		\dfrac{-58\pi}{3} &= \dfrac{-58}{3}\times \dfrac{2\pi}{2} \\
		&=\dfrac{-58}{6} \times 2 \pi\\
		&=\dfrac{-(6\times 9+4) \times 2 \pi}{6}\\
		&=-\dfrac{6\times 9 \times 2 \pi}{6}-\dfrac{4\times 2\pi}{6}\\
		&=-9\times 2\pi-\dfrac{4\pi}{3}
	\end{align*}
\end{minipage}
\hfil
\begin{minipage}{0.5 \linewidth}
	\begin{tikzpicture}[scale = 0.65]
		\draw[thick] (0,0) circle (2);
		\draw[-{Straight Barb[length = 0.5mm]}] (-2.25,0) -- (2.25, 0);
		\draw[-{Straight Barb[length = 0.5mm]}] (0,-2.25) -- (0, 2.25);
		\begin{scope}[rotate = 240.0]
	\draw[dotted] (0,0) -- (2,0);
	\draw[thick] (1.9, 0) -- (2.1,0);
	\end{scope}

\begin{scope}[rotate = 300.0]
	\draw[dotted] (0,0) -- (2,0);
	\draw[thick] (1.9, 0) -- (2.1,0);
	\end{scope}

\begin{scope}[rotate = 120.0]
	\draw[dotted, louis, thick] (0,0) -- (2,0);
	\draw[louis, thick] (1.9, 0) -- (2.1,0);
	\draw[louis] (2.3, 0) node [above, left] {$\dfrac{-58\pi}{3}=\dfrac{2\pi}{3}$};
\end{scope}

\draw[dashed, louis, thick] ({2*cos(deg 2/3*pi)},0) -- ({2*cos(deg 2/3*pi)}, {2*sin(deg 2/3*pi)}) -- (0, {2*sin(deg 2/3*pi)});\begin{scope}[rotate = 60.0]
	\draw[thick, dotted, louis] (0,0) -- (2,0);
	\draw[thick, louis] (1.9, 0) -- (2.1,0) node[above right] {$\dfrac{\pi}{3}$};
\end{scope}

\draw[dashed, louis, thick] ({2*cos(deg 1/3*pi)},0) -- ({2*cos(deg 1/3*pi)}, {2*sin(deg 1/3*pi)}) -- (0, {2*sin(deg 1/3*pi)});\end{tikzpicture}
\end{minipage}

Or $-\dfrac{4\pi}{3}\leq-\pi$, on fait un tour de plus en rajoutant $2\pi$: $-\dfrac{4\pi}{3}+2\pi = \dfrac{2\pi}{3}$.

Comme $-\pi <\dfrac{2\pi}{3}\leq \pi$, la mesure principale de $\dfrac{-58\pi}{3}$ est $\dfrac{2\pi}{3}$.\end{frame}


\begin{frame}
\vspace{-10mm}
	\frametitle{Correction 8}
\begin{minipage}{0.45 \linewidth}
	\begin{align*}
		\dfrac{49\pi}{5} &= \dfrac{49}{5}\times \dfrac{2\pi}{2} \\
		&=\dfrac{49}{10} \times 2 \pi\\
		&=\dfrac{(10\times 4+9) \times 2 \pi}{10}\\
		&=\dfrac{10\times 4 \times 2 \pi}{10}+\dfrac{9\times 2\pi}{10}\\
		&=4\times 2\pi+\dfrac{9\pi}{5}
	\end{align*}
\end{minipage}
\hfil
\begin{minipage}{0.5 \linewidth}
	\begin{tikzpicture}[scale = 0.65]
		\draw[thick] (0,0) circle (2);
		\draw[-{Straight Barb[length = 0.5mm]}] (-2.25,0) -- (2.25, 0);
		\draw[-{Straight Barb[length = 0.5mm]}] (0,-2.25) -- (0, 2.25);
		\begin{scope}[rotate = 72.0]
	\draw[dotted] (0,0) -- (2,0);
	\draw[thick] (1.9, 0) -- (2.1,0);
	\end{scope}

\begin{scope}[rotate = 108.0]
	\draw[dotted] (0,0) -- (2,0);
	\draw[thick] (1.9, 0) -- (2.1,0);
	\end{scope}

\begin{scope}[rotate = 144.0]
	\draw[dotted] (0,0) -- (2,0);
	\draw[thick] (1.9, 0) -- (2.1,0);
	\end{scope}

\begin{scope}[rotate = 216.0]
	\draw[dotted] (0,0) -- (2,0);
	\draw[thick] (1.9, 0) -- (2.1,0);
	\end{scope}

\begin{scope}[rotate = 252.0]
	\draw[dotted] (0,0) -- (2,0);
	\draw[thick] (1.9, 0) -- (2.1,0);
	\end{scope}

\begin{scope}[rotate = 288.0]
	\draw[dotted] (0,0) -- (2,0);
	\draw[thick] (1.9, 0) -- (2.1,0);
	\end{scope}

\begin{scope}[rotate = -36.0]
	\draw[dotted, louis, thick] (0,0) -- (2,0);
	\draw[louis, thick] (1.9, 0) -- (2.1,0);
	\draw[louis] (2.3, 0) node [below, right] {$\dfrac{49\pi}{5}=\dfrac{-\pi}{5}$};
\end{scope}

\draw[dashed, louis, thick] ({2*cos(deg -1/5*pi)},0) -- ({2*cos(deg -1/5*pi)}, {2*sin(deg -1/5*pi)}) -- (0, {2*sin(deg -1/5*pi)});\begin{scope}[rotate = 36.0]
	\draw[thick, dotted, louis] (0,0) -- (2,0);
	\draw[thick, louis] (1.9, 0) -- (2.1,0) node[above right] {$\dfrac{\pi}{5}$};
\end{scope}

\draw[dashed, louis, thick] ({2*cos(deg 1/5*pi)},0) -- ({2*cos(deg 1/5*pi)}, {2*sin(deg 1/5*pi)}) -- (0, {2*sin(deg 1/5*pi)});\end{tikzpicture}
\end{minipage}

Or $\dfrac{9\pi}{5}>\pi$, on fait un tour de moins en retirant $2\pi$: $\dfrac{9\pi}{5}-2\pi = \dfrac{-\pi}{5}$.

Comme $-\pi <-\dfrac{\pi}{5}\leq \pi$, la mesure principale de $\dfrac{49\pi}{5}$ est $\dfrac{-\pi}{5}$.\end{frame}


\begin{frame}
\vspace{-10mm}
	\frametitle{Correction 9}
\begin{minipage}{0.45 \linewidth}
	\begin{align*}
		\dfrac{104\pi}{5} &= \dfrac{104}{5}\times \dfrac{2\pi}{2} \\
		&=\dfrac{104}{10} \times 2 \pi\\
		&=\dfrac{(10\times 10+4) \times 2 \pi}{10}\\
		&=\dfrac{10\times 10 \times 2 \pi}{10}+\dfrac{4\times 2\pi}{10}\\
		&=10\times 2\pi+\dfrac{4\pi}{5}
	\end{align*}
\end{minipage}
\hfil
\begin{minipage}{0.5 \linewidth}
	\begin{tikzpicture}[scale = 0.65]
		\draw[thick] (0,0) circle (2);
		\draw[-{Straight Barb[length = 0.5mm]}] (-2.25,0) -- (2.25, 0);
		\draw[-{Straight Barb[length = 0.5mm]}] (0,-2.25) -- (0, 2.25);
		\begin{scope}[rotate = 72.0]
	\draw[dotted] (0,0) -- (2,0);
	\draw[thick] (1.9, 0) -- (2.1,0);
	\end{scope}

\begin{scope}[rotate = 108.0]
	\draw[dotted] (0,0) -- (2,0);
	\draw[thick] (1.9, 0) -- (2.1,0);
	\end{scope}

\begin{scope}[rotate = 216.0]
	\draw[dotted] (0,0) -- (2,0);
	\draw[thick] (1.9, 0) -- (2.1,0);
	\end{scope}

\begin{scope}[rotate = 252.0]
	\draw[dotted] (0,0) -- (2,0);
	\draw[thick] (1.9, 0) -- (2.1,0);
	\end{scope}

\begin{scope}[rotate = 288.0]
	\draw[dotted] (0,0) -- (2,0);
	\draw[thick] (1.9, 0) -- (2.1,0);
	\end{scope}

\begin{scope}[rotate = 324.0]
	\draw[dotted] (0,0) -- (2,0);
	\draw[thick] (1.9, 0) -- (2.1,0);
	\end{scope}

\begin{scope}[rotate = 144.0]
	\draw[dotted, louis, thick] (0,0) -- (2,0);
	\draw[louis, thick] (1.9, 0) -- (2.1,0);
	\draw[louis] (2.3, 0) node [above, left] {$\dfrac{104\pi}{5}=\dfrac{4\pi}{5}$};
\end{scope}

\draw[dashed, louis, thick] ({2*cos(deg 4/5*pi)},0) -- ({2*cos(deg 4/5*pi)}, {2*sin(deg 4/5*pi)}) -- (0, {2*sin(deg 4/5*pi)});\begin{scope}[rotate = 36.0]
	\draw[thick, dotted, louis] (0,0) -- (2,0);
	\draw[thick, louis] (1.9, 0) -- (2.1,0) node[above right] {$\dfrac{\pi}{5}$};
\end{scope}

\draw[dashed, louis, thick] ({2*cos(deg 1/5*pi)},0) -- ({2*cos(deg 1/5*pi)}, {2*sin(deg 1/5*pi)}) -- (0, {2*sin(deg 1/5*pi)});\end{tikzpicture}
\end{minipage}

Comme $-\pi < \dfrac{4\pi}{5}\leq \pi$, la mesure principale de $\dfrac{104\pi}{5}$ est $\dfrac{-4\pi}{5}$.\end{frame}


\begin{frame}
\vspace{-10mm}
	\frametitle{Correction 10}
\begin{minipage}{0.45 \linewidth}
	\begin{align*}
		\dfrac{-12\pi}{5} &= \dfrac{-12}{5}\times \dfrac{2\pi}{2} \\
		&=\dfrac{-12}{10} \times 2 \pi\\
		&=\dfrac{-(10\times 1+2) \times 2 \pi}{10}\\
		&=-\dfrac{10\times 1 \times 2 \pi}{10}-\dfrac{2\times 2\pi}{10}\\
		&=-1\times 2\pi-\dfrac{2\pi}{5}
	\end{align*}
\end{minipage}
\hfil
\begin{minipage}{0.5 \linewidth}
	\begin{tikzpicture}[scale = 0.65]
		\draw[thick] (0,0) circle (2);
		\draw[-{Straight Barb[length = 0.5mm]}] (-2.25,0) -- (2.25, 0);
		\draw[-{Straight Barb[length = 0.5mm]}] (0,-2.25) -- (0, 2.25);
		\begin{scope}[rotate = 72.0]
	\draw[dotted] (0,0) -- (2,0);
	\draw[thick] (1.9, 0) -- (2.1,0);
	\end{scope}

\begin{scope}[rotate = 108.0]
	\draw[dotted] (0,0) -- (2,0);
	\draw[thick] (1.9, 0) -- (2.1,0);
	\end{scope}

\begin{scope}[rotate = 144.0]
	\draw[dotted] (0,0) -- (2,0);
	\draw[thick] (1.9, 0) -- (2.1,0);
	\end{scope}

\begin{scope}[rotate = 216.0]
	\draw[dotted] (0,0) -- (2,0);
	\draw[thick] (1.9, 0) -- (2.1,0);
	\end{scope}

\begin{scope}[rotate = 252.0]
	\draw[dotted] (0,0) -- (2,0);
	\draw[thick] (1.9, 0) -- (2.1,0);
	\end{scope}

\begin{scope}[rotate = 324.0]
	\draw[dotted] (0,0) -- (2,0);
	\draw[thick] (1.9, 0) -- (2.1,0);
	\end{scope}

\begin{scope}[rotate = -72.0]
	\draw[dotted, louis, thick] (0,0) -- (2,0);
	\draw[louis, thick] (1.9, 0) -- (2.1,0);
	\draw[louis] (2.3, 0) node [below, right] {$\dfrac{-12\pi}{5}=\dfrac{-2\pi}{5}$};
\end{scope}

\draw[dashed, louis, thick] ({2*cos(deg -2/5*pi)},0) -- ({2*cos(deg -2/5*pi)}, {2*sin(deg -2/5*pi)}) -- (0, {2*sin(deg -2/5*pi)});\begin{scope}[rotate = 36.0]
	\draw[thick, dotted, louis] (0,0) -- (2,0);
	\draw[thick, louis] (1.9, 0) -- (2.1,0) node[above right] {$\dfrac{\pi}{5}$};
\end{scope}

\draw[dashed, louis, thick] ({2*cos(deg 1/5*pi)},0) -- ({2*cos(deg 1/5*pi)}, {2*sin(deg 1/5*pi)}) -- (0, {2*sin(deg 1/5*pi)});\end{tikzpicture}
\end{minipage}

Comme $-\pi < -\dfrac{2\pi}{5}\leq \pi$, la mesure principale de $\dfrac{-12\pi}{5}$ est $\dfrac{-2\pi}{5}$.\end{frame}


\begin{frame}
\vspace{-10mm}
	\frametitle{Correction 11}
\begin{minipage}{0.45 \linewidth}
	\begin{align*}
		\dfrac{131\pi}{4} &= \dfrac{131}{4}\times \dfrac{2\pi}{2} \\
		&=\dfrac{131}{8} \times 2 \pi\\
		&=\dfrac{(8\times 16+3) \times 2 \pi}{8}\\
		&=\dfrac{8\times 16 \times 2 \pi}{8}+\dfrac{3\times 2\pi}{8}\\
		&=16\times 2\pi+\dfrac{3\pi}{4}
	\end{align*}
\end{minipage}
\hfil
\begin{minipage}{0.5 \linewidth}
	\begin{tikzpicture}[scale = 0.65]
		\draw[thick] (0,0) circle (2);
		\draw[-{Straight Barb[length = 0.5mm]}] (-2.25,0) -- (2.25, 0);
		\draw[-{Straight Barb[length = 0.5mm]}] (0,-2.25) -- (0, 2.25);
		\begin{scope}[rotate = 90.0]
	\draw[dotted] (0,0) -- (2,0);
	\draw[thick] (1.9, 0) -- (2.1,0);
	\end{scope}

\begin{scope}[rotate = 225.0]
	\draw[dotted] (0,0) -- (2,0);
	\draw[thick] (1.9, 0) -- (2.1,0);
	\end{scope}

\begin{scope}[rotate = 270.0]
	\draw[dotted] (0,0) -- (2,0);
	\draw[thick] (1.9, 0) -- (2.1,0);
	\end{scope}

\begin{scope}[rotate = 315.0]
	\draw[dotted] (0,0) -- (2,0);
	\draw[thick] (1.9, 0) -- (2.1,0);
	\end{scope}

\begin{scope}[rotate = 135.0]
	\draw[dotted, louis, thick] (0,0) -- (2,0);
	\draw[louis, thick] (1.9, 0) -- (2.1,0);
	\draw[louis] (2.3, 0) node [above, left] {$\dfrac{131\pi}{4}=\dfrac{3\pi}{4}$};
\end{scope}

\draw[dashed, louis, thick] ({2*cos(deg 3/4*pi)},0) -- ({2*cos(deg 3/4*pi)}, {2*sin(deg 3/4*pi)}) -- (0, {2*sin(deg 3/4*pi)});\begin{scope}[rotate = 45.0]
	\draw[thick, dotted, louis] (0,0) -- (2,0);
	\draw[thick, louis] (1.9, 0) -- (2.1,0) node[above right] {$\dfrac{\pi}{4}$};
\end{scope}

\draw[dashed, louis, thick] ({2*cos(deg 1/4*pi)},0) -- ({2*cos(deg 1/4*pi)}, {2*sin(deg 1/4*pi)}) -- (0, {2*sin(deg 1/4*pi)});\end{tikzpicture}
\end{minipage}

Comme $-\pi < \dfrac{3\pi}{4}\leq \pi$, la mesure principale de $\dfrac{131\pi}{4}$ est $\dfrac{-3\pi}{4}$.\end{frame}


\begin{frame}
\vspace{-10mm}
	\frametitle{Correction 12}
\begin{minipage}{0.45 \linewidth}
	\begin{align*}
		\dfrac{-119\pi}{11} &= \dfrac{-119}{11}\times \dfrac{2\pi}{2} \\
		&=\dfrac{-119}{22} \times 2 \pi\\
		&=\dfrac{-(22\times 5+9) \times 2 \pi}{22}\\
		&=-\dfrac{22\times 5 \times 2 \pi}{22}-\dfrac{9\times 2\pi}{22}\\
		&=-5\times 2\pi-\dfrac{9\pi}{11}
	\end{align*}
\end{minipage}
\hfil
\begin{minipage}{0.5 \linewidth}
	\begin{tikzpicture}[scale = 0.65]
		\draw[thick] (0,0) circle (2);
		\draw[-{Straight Barb[length = 0.5mm]}] (-2.25,0) -- (2.25, 0);
		\draw[-{Straight Barb[length = 0.5mm]}] (0,-2.25) -- (0, 2.25);
		\begin{scope}[rotate = 32.72]
	\draw[dotted] (0,0) -- (2,0);
	\draw[thick] (1.9, 0) -- (2.1,0);
	\end{scope}

\begin{scope}[rotate = 49.08]
	\draw[dotted] (0,0) -- (2,0);
	\draw[thick] (1.9, 0) -- (2.1,0);
	\end{scope}

\begin{scope}[rotate = 65.44]
	\draw[dotted] (0,0) -- (2,0);
	\draw[thick] (1.9, 0) -- (2.1,0);
	\end{scope}

\begin{scope}[rotate = 81.8]
	\draw[dotted] (0,0) -- (2,0);
	\draw[thick] (1.9, 0) -- (2.1,0);
	\end{scope}

\begin{scope}[rotate = 98.16]
	\draw[dotted] (0,0) -- (2,0);
	\draw[thick] (1.9, 0) -- (2.1,0);
	\end{scope}

\begin{scope}[rotate = 114.52]
	\draw[dotted] (0,0) -- (2,0);
	\draw[thick] (1.9, 0) -- (2.1,0);
	\end{scope}

\begin{scope}[rotate = 130.88]
	\draw[dotted] (0,0) -- (2,0);
	\draw[thick] (1.9, 0) -- (2.1,0);
	\end{scope}

\begin{scope}[rotate = 147.24]
	\draw[dotted] (0,0) -- (2,0);
	\draw[thick] (1.9, 0) -- (2.1,0);
	\end{scope}

\begin{scope}[rotate = 163.6]
	\draw[dotted] (0,0) -- (2,0);
	\draw[thick] (1.9, 0) -- (2.1,0);
	\end{scope}

\begin{scope}[rotate = 179.95999999999998]
	\draw[dotted] (0,0) -- (2,0);
	\draw[thick] (1.9, 0) -- (2.1,0);
	\end{scope}

\begin{scope}[rotate = 196.32]
	\draw[dotted] (0,0) -- (2,0);
	\draw[thick] (1.9, 0) -- (2.1,0);
	\end{scope}

\begin{scope}[rotate = 229.04]
	\draw[dotted] (0,0) -- (2,0);
	\draw[thick] (1.9, 0) -- (2.1,0);
	\end{scope}

\begin{scope}[rotate = 245.39999999999998]
	\draw[dotted] (0,0) -- (2,0);
	\draw[thick] (1.9, 0) -- (2.1,0);
	\end{scope}

\begin{scope}[rotate = 261.76]
	\draw[dotted] (0,0) -- (2,0);
	\draw[thick] (1.9, 0) -- (2.1,0);
	\end{scope}

\begin{scope}[rotate = 278.12]
	\draw[dotted] (0,0) -- (2,0);
	\draw[thick] (1.9, 0) -- (2.1,0);
	\end{scope}

\begin{scope}[rotate = 294.48]
	\draw[dotted] (0,0) -- (2,0);
	\draw[thick] (1.9, 0) -- (2.1,0);
	\end{scope}

\begin{scope}[rotate = 310.84]
	\draw[dotted] (0,0) -- (2,0);
	\draw[thick] (1.9, 0) -- (2.1,0);
	\end{scope}

\begin{scope}[rotate = 327.2]
	\draw[dotted] (0,0) -- (2,0);
	\draw[thick] (1.9, 0) -- (2.1,0);
	\end{scope}

\begin{scope}[rotate = 343.56]
	\draw[dotted] (0,0) -- (2,0);
	\draw[thick] (1.9, 0) -- (2.1,0);
	\end{scope}

\begin{scope}[rotate = -147.24]
	\draw[dotted, louis, thick] (0,0) -- (2,0);
	\draw[louis, thick] (1.9, 0) -- (2.1,0);
	\draw[louis] (2.3, 0) node [below, left] {$\dfrac{-119\pi}{11}=\dfrac{-9\pi}{11}$};
\end{scope}

\draw[dashed, louis, thick] ({2*cos(deg -9/11*pi)},0) -- ({2*cos(deg -9/11*pi)}, {2*sin(deg -9/11*pi)}) -- (0, {2*sin(deg -9/11*pi)});\begin{scope}[rotate = 16.36]
	\draw[thick, dotted, louis] (0,0) -- (2,0);
	\draw[thick, louis] (1.9, 0) -- (2.1,0) node[above right] {$\dfrac{\pi}{11}$};
\end{scope}

\draw[dashed, louis, thick] ({2*cos(deg 1/11*pi)},0) -- ({2*cos(deg 1/11*pi)}, {2*sin(deg 1/11*pi)}) -- (0, {2*sin(deg 1/11*pi)});\end{tikzpicture}
\end{minipage}

Comme $-\pi < -\dfrac{9\pi}{11}\leq \pi$, la mesure principale de $\dfrac{-119\pi}{11}$ est $\dfrac{-9\pi}{11}$.\end{frame}


\begin{frame}
\vspace{-10mm}
	\frametitle{Correction 13}
\begin{minipage}{0.45 \linewidth}
	\begin{align*}
		\dfrac{-105\pi}{11} &= \dfrac{-105}{11}\times \dfrac{2\pi}{2} \\
		&=\dfrac{-105}{22} \times 2 \pi\\
		&=\dfrac{-(22\times 4+17) \times 2 \pi}{22}\\
		&=-\dfrac{22\times 4 \times 2 \pi}{22}-\dfrac{17\times 2\pi}{22}\\
		&=-4\times 2\pi-\dfrac{17\pi}{11}
	\end{align*}
\end{minipage}
\hfil
\begin{minipage}{0.5 \linewidth}
	\begin{tikzpicture}[scale = 0.65]
		\draw[thick] (0,0) circle (2);
		\draw[-{Straight Barb[length = 0.5mm]}] (-2.25,0) -- (2.25, 0);
		\draw[-{Straight Barb[length = 0.5mm]}] (0,-2.25) -- (0, 2.25);
		\begin{scope}[rotate = 32.72]
	\draw[dotted] (0,0) -- (2,0);
	\draw[thick] (1.9, 0) -- (2.1,0);
	\end{scope}

\begin{scope}[rotate = 49.08]
	\draw[dotted] (0,0) -- (2,0);
	\draw[thick] (1.9, 0) -- (2.1,0);
	\end{scope}

\begin{scope}[rotate = 65.44]
	\draw[dotted] (0,0) -- (2,0);
	\draw[thick] (1.9, 0) -- (2.1,0);
	\end{scope}

\begin{scope}[rotate = 98.16]
	\draw[dotted] (0,0) -- (2,0);
	\draw[thick] (1.9, 0) -- (2.1,0);
	\end{scope}

\begin{scope}[rotate = 114.52]
	\draw[dotted] (0,0) -- (2,0);
	\draw[thick] (1.9, 0) -- (2.1,0);
	\end{scope}

\begin{scope}[rotate = 130.88]
	\draw[dotted] (0,0) -- (2,0);
	\draw[thick] (1.9, 0) -- (2.1,0);
	\end{scope}

\begin{scope}[rotate = 147.24]
	\draw[dotted] (0,0) -- (2,0);
	\draw[thick] (1.9, 0) -- (2.1,0);
	\end{scope}

\begin{scope}[rotate = 163.6]
	\draw[dotted] (0,0) -- (2,0);
	\draw[thick] (1.9, 0) -- (2.1,0);
	\end{scope}

\begin{scope}[rotate = 179.95999999999998]
	\draw[dotted] (0,0) -- (2,0);
	\draw[thick] (1.9, 0) -- (2.1,0);
	\end{scope}

\begin{scope}[rotate = 196.32]
	\draw[dotted] (0,0) -- (2,0);
	\draw[thick] (1.9, 0) -- (2.1,0);
	\end{scope}

\begin{scope}[rotate = 212.68]
	\draw[dotted] (0,0) -- (2,0);
	\draw[thick] (1.9, 0) -- (2.1,0);
	\end{scope}

\begin{scope}[rotate = 229.04]
	\draw[dotted] (0,0) -- (2,0);
	\draw[thick] (1.9, 0) -- (2.1,0);
	\end{scope}

\begin{scope}[rotate = 245.39999999999998]
	\draw[dotted] (0,0) -- (2,0);
	\draw[thick] (1.9, 0) -- (2.1,0);
	\end{scope}

\begin{scope}[rotate = 261.76]
	\draw[dotted] (0,0) -- (2,0);
	\draw[thick] (1.9, 0) -- (2.1,0);
	\end{scope}

\begin{scope}[rotate = 278.12]
	\draw[dotted] (0,0) -- (2,0);
	\draw[thick] (1.9, 0) -- (2.1,0);
	\end{scope}

\begin{scope}[rotate = 294.48]
	\draw[dotted] (0,0) -- (2,0);
	\draw[thick] (1.9, 0) -- (2.1,0);
	\end{scope}

\begin{scope}[rotate = 310.84]
	\draw[dotted] (0,0) -- (2,0);
	\draw[thick] (1.9, 0) -- (2.1,0);
	\end{scope}

\begin{scope}[rotate = 327.2]
	\draw[dotted] (0,0) -- (2,0);
	\draw[thick] (1.9, 0) -- (2.1,0);
	\end{scope}

\begin{scope}[rotate = 343.56]
	\draw[dotted] (0,0) -- (2,0);
	\draw[thick] (1.9, 0) -- (2.1,0);
	\end{scope}

\begin{scope}[rotate = 81.8]
	\draw[dotted, louis, thick] (0,0) -- (2,0);
	\draw[louis, thick] (1.9, 0) -- (2.1,0);
	\draw[louis] (2.3, 0) node [above, right] {$\dfrac{-105\pi}{11}=\dfrac{5\pi}{11}$};
\end{scope}

\draw[dashed, louis, thick] ({2*cos(deg 5/11*pi)},0) -- ({2*cos(deg 5/11*pi)}, {2*sin(deg 5/11*pi)}) -- (0, {2*sin(deg 5/11*pi)});\begin{scope}[rotate = 16.36]
	\draw[thick, dotted, louis] (0,0) -- (2,0);
	\draw[thick, louis] (1.9, 0) -- (2.1,0) node[above right] {$\dfrac{\pi}{11}$};
\end{scope}

\draw[dashed, louis, thick] ({2*cos(deg 1/11*pi)},0) -- ({2*cos(deg 1/11*pi)}, {2*sin(deg 1/11*pi)}) -- (0, {2*sin(deg 1/11*pi)});\end{tikzpicture}
\end{minipage}

Or $-\dfrac{17\pi}{11}\leq-\pi$, on fait un tour de plus en rajoutant $2\pi$: $-\dfrac{17\pi}{11}+2\pi = \dfrac{5\pi}{11}$.

Comme $-\pi <\dfrac{5\pi}{11}\leq \pi$, la mesure principale de $\dfrac{-105\pi}{11}$ est $\dfrac{5\pi}{11}$.\end{frame}


\begin{frame}
\vspace{-10mm}
	\frametitle{Correction 14}
\begin{minipage}{0.45 \linewidth}
	\begin{align*}
		\dfrac{-118\pi}{7} &= \dfrac{-118}{7}\times \dfrac{2\pi}{2} \\
		&=\dfrac{-118}{14} \times 2 \pi\\
		&=\dfrac{-(14\times 8+6) \times 2 \pi}{14}\\
		&=-\dfrac{14\times 8 \times 2 \pi}{14}-\dfrac{6\times 2\pi}{14}\\
		&=-8\times 2\pi-\dfrac{6\pi}{7}
	\end{align*}
\end{minipage}
\hfil
\begin{minipage}{0.5 \linewidth}
	\begin{tikzpicture}[scale = 0.65]
		\draw[thick] (0,0) circle (2);
		\draw[-{Straight Barb[length = 0.5mm]}] (-2.25,0) -- (2.25, 0);
		\draw[-{Straight Barb[length = 0.5mm]}] (0,-2.25) -- (0, 2.25);
		\begin{scope}[rotate = 51.42]
	\draw[dotted] (0,0) -- (2,0);
	\draw[thick] (1.9, 0) -- (2.1,0);
	\end{scope}

\begin{scope}[rotate = 77.13]
	\draw[dotted] (0,0) -- (2,0);
	\draw[thick] (1.9, 0) -- (2.1,0);
	\end{scope}

\begin{scope}[rotate = 102.84]
	\draw[dotted] (0,0) -- (2,0);
	\draw[thick] (1.9, 0) -- (2.1,0);
	\end{scope}

\begin{scope}[rotate = 128.55]
	\draw[dotted] (0,0) -- (2,0);
	\draw[thick] (1.9, 0) -- (2.1,0);
	\end{scope}

\begin{scope}[rotate = 154.26]
	\draw[dotted] (0,0) -- (2,0);
	\draw[thick] (1.9, 0) -- (2.1,0);
	\end{scope}

\begin{scope}[rotate = 179.97]
	\draw[dotted] (0,0) -- (2,0);
	\draw[thick] (1.9, 0) -- (2.1,0);
	\end{scope}

\begin{scope}[rotate = 231.39000000000001]
	\draw[dotted] (0,0) -- (2,0);
	\draw[thick] (1.9, 0) -- (2.1,0);
	\end{scope}

\begin{scope}[rotate = 257.1]
	\draw[dotted] (0,0) -- (2,0);
	\draw[thick] (1.9, 0) -- (2.1,0);
	\end{scope}

\begin{scope}[rotate = 282.81]
	\draw[dotted] (0,0) -- (2,0);
	\draw[thick] (1.9, 0) -- (2.1,0);
	\end{scope}

\begin{scope}[rotate = 308.52]
	\draw[dotted] (0,0) -- (2,0);
	\draw[thick] (1.9, 0) -- (2.1,0);
	\end{scope}

\begin{scope}[rotate = 334.23]
	\draw[dotted] (0,0) -- (2,0);
	\draw[thick] (1.9, 0) -- (2.1,0);
	\end{scope}

\begin{scope}[rotate = -154.26]
	\draw[dotted, louis, thick] (0,0) -- (2,0);
	\draw[louis, thick] (1.9, 0) -- (2.1,0);
	\draw[louis] (2.3, 0) node [below, left] {$\dfrac{-118\pi}{7}=\dfrac{-6\pi}{7}$};
\end{scope}

\draw[dashed, louis, thick] ({2*cos(deg -6/7*pi)},0) -- ({2*cos(deg -6/7*pi)}, {2*sin(deg -6/7*pi)}) -- (0, {2*sin(deg -6/7*pi)});\begin{scope}[rotate = 25.71]
	\draw[thick, dotted, louis] (0,0) -- (2,0);
	\draw[thick, louis] (1.9, 0) -- (2.1,0) node[above right] {$\dfrac{\pi}{7}$};
\end{scope}

\draw[dashed, louis, thick] ({2*cos(deg 1/7*pi)},0) -- ({2*cos(deg 1/7*pi)}, {2*sin(deg 1/7*pi)}) -- (0, {2*sin(deg 1/7*pi)});\end{tikzpicture}
\end{minipage}

Comme $-\pi < -\dfrac{6\pi}{7}\leq \pi$, la mesure principale de $\dfrac{-118\pi}{7}$ est $\dfrac{-6\pi}{7}$.\end{frame}




\end{document}