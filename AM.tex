\documentclass[15pt, mathserif]{beamer}

\usepackage[french]{babel}
\usepackage[T1]{fontenc}
\usepackage[utf8]{inputenc}
%\usepackage{esvect}
\usepackage{bm}
\usepackage{eurosym}
\usepackage{tikz}
\usepackage{pgf,tikz,pgfplots}
\pgfplotsset{compat=1.15}
\usepackage{mathrsfs}
\usetikzlibrary{arrows}
\usetikzlibrary{arrows.meta}

\usetikzlibrary{mindmap}
\usepackage{multicol}
\usepackage[tikz]{bclogo}
\usepackage{tkz-tab}
\usepackage{amsmath, tabu}
\usepackage{esvect} %\vv{AB} pour le vecteur AB
%% Tableau

\usepackage{makecell}
\setcellgapes{1pt}
\makegapedcells
\newcolumntype{R}[1]{>{\raggedleft\arraybackslash }b{#1}}
\newcolumntype{L}[1]{>{\raggedright\arraybackslash }b{#1}}
\newcolumntype{C}[1]{>{\centering\arraybackslash }b{#1}}


%pour avoir des parenthèses rondes dans le package fourier
\DeclareSymbolFont{cmoperators}   {OT1}{cmr} {m}{n}
\DeclareSymbolFont{cmlargesymbols}{OMX}{cmex}{m}{n}

\usefonttheme{professionalfonts} %permet d'enlever un bug avec fourier
\usepackage{fourier}
\DeclareMathDelimiter{(}{\mathopen} {cmoperators}{"28}{cmlargesymbols}{"00}
\DeclareMathDelimiter{)}{\mathclose}{cmoperators}{"29}{cmlargesymbols}{"01}

%Graphiques 

\usepackage{pgf,tikz,pgfplots}
\pgfplotsset{compat=1.15}
\usepackage{mathrsfs}
\usetikzlibrary{arrows}
\usetikzlibrary{mindmap}

%ensembles de nbres

\newcommand{\R}{\mathbb{R}}			%permet d'écrire le R "ensemble des réels"'
\newcommand{\N}{\mathbb{N}}			%permet d'écrire le N "ensemble des entiers naturels"
\newcommand{\Z}{\mathbb{Z}}			%permet d'écrire le Z "ensemble des entiers relatifs"
\newcommand{\Prem}{\mathbb{P}}	%permet d'écrire le P "ensemble des nombres premiers" (qui n'a pas marché avec le \P car il existe déjà)
\newcommand{\D}{\mathbb{D}}
\newcommand{\Df}{\mathcal{D}_f}
\newcommand{\Cf}{\mathcal{C}_f}

\newcommand{\Q}{\mathbb{Q}}

\usetheme{Madrid}
\useoutertheme{miniframes} % Alternatively: miniframes, infolines, split
\useinnertheme{circles}
\definecolor{UBCblue}{rgb}{0.1, 0.25, 0.4} % UBC Blue (primary)
\definecolor{bordeaux}{RGB}{128,0,0}
\usecolortheme[named=UBCblue]{structure}

\usepackage{color} % J'aime bien définir mes couleurs
\definecolor{propcolor}{rgb}{0, 0.5, 1}
\colorlet{louis}{blue!70!green!60!white}
\colorlet{sakura}{pink!40!red}

\title{Activités Mentales}
\date{03 Septembre 2022}

\newcommand{\vco}[2]{\begin{pmatrix} #1 \\ #2 \end{pmatrix}} %Coordonnées de vecteur
\newenvironment{eq}{\begin{cases}\begin{tabu}{ccccc}}{\end{tabu}\end{cases}}
\newenvironment{eql}{\begin{cases}\begin{tabu}{cccccl}}{\end{tabu}\end{cases}}
\newenvironment{eqrl}{\begin{cases}\begin{tabu}{rl}}{\end{tabu}\end{cases}}


\begin{document}

\begin{frame}
    \titlepage
\end{frame}

\begin{frame} 
	\frametitle{Question 1}
Quelle est l'expression de la fonction affine passant par les points de coordonnées (7;62) et (9;82) ?\end{frame}


\begin{frame} 
	\frametitle{Question 2}
	Écrire sous la forme $a^n$, où $a$ et $n$ sont des entiers relatifs, le nombre suivant \[\left( 2^{9} \right) ^{ 10}. \]\end{frame}


\begin{frame} 
	\frametitle{Question 3}
Écrire sous la forme $a^n$, où $a$ et $n$ sont des entiers relatifs, le nombre suivant \[4^{-13} \times 4^{14}. \]\end{frame}


\begin{frame} 
	\frametitle{Question 4}
Quelle est l'expression de la fonction affine passant par les points de coordonnées (8;-64) et (-4;32) ?\end{frame}


\begin{frame} 
	\frametitle{Question 5}
Quelle est l'expression de la fonction affine passant par les points de coordonnées (0;0) et (1;4) ?\end{frame}


\begin{frame} 
	\frametitle{Question 6}
	Écrire sous la forme $a^n$, où $a$ et $n$ sont des entiers relatifs, le nombre suivant \[\left( 3^{-9} \right) ^{ 10}. \]\end{frame}


\begin{frame}
\vspace{-10mm}
	\frametitle{Correction 1}
\vspace*{1cm} 
 \footnotesize{Quelle est l'expression de la fonction affine passant par les points de coordonnées (7;62) et (9;82) ? Il existe deux techniques :} 
 \begin{multicols}{2} 
 \begin{enumerate} 
 \item On résout un système : $$ \begin{array}{rcl} 
 & \textcolor{white}{\Leftrightarrow} & 
 \left 
 \{\begin{array}{rcl}7\times m + p&=&62 \\ 
 9\times m+p&=&82\end{array} \right. \\ 
 &\Leftrightarrow & \left 
 \{\begin{array}{rcl} p&=&62-7m \\ 
 9m+p&=&82\end{array} \right. \\ 
 &\Leftrightarrow & \left 
 \{\begin{array}{rcl} p&=&62-7m \\ 
 9m+(62-7m) &=&82\end{array} \right. \\ &\Leftrightarrow& \left \{\begin{array}{rcl}p&=&62-7m \\ 
 62+2m&=&82\end{array} \right. \\ &\Leftrightarrow& \left \{\begin{array}{rcl}p&=&62-7m \\ 
 2m&=&20\end{array} \right. \\  &\Leftrightarrow& \left \{\begin{array}{rcl} p&=&-8 \\  m&=&10\end{array}\right. \end{array}$$ 
 Ainsi on a $f:x\mapsto 10x-8$ 
 \columnbreak 
 \item 
 \footnotesize{On applique la formule du cours pour calculer $m$ :$$ \dfrac{f(x_1)-f(x_2)}{x_1-x_2}=\dfrac{62-82}{7-9}= \dfrac{-20}{-2}=10$$} \footnotesize{ Ainsi on a $f(x)= 10x +p $. 
  \\ On cherche maintenant la valeur de $p$. On sait que $f(7)=62$. On doit donc résoudre $(E): 10\times7+p=62$}	 
 \begin{align*} (E)& \Leftrightarrow 70+p=62\\
		 	 & \Leftrightarrow p=62-70\\
			 & \Leftrightarrow p=-8
	 \end{align*} 
 Ainsi on a $f:x\mapsto 10x-8$ 
 \end{enumerate} 
 \end{multicols} 
 \end{frame}


\begin{frame}
\vspace{-10mm}
	\frametitle{Correction 2}
\[\left( 2^{9}\right)^{10} =2^{9 \times 10} = 2^{90}\]
\end{frame}


\begin{frame}
\vspace{-10mm}
	\frametitle{Correction 3}
\[4^{-13} \times 4^{14} = 4^{-13+14} = 4^{1} = 4\]\end{frame}


\begin{frame}
\vspace{-10mm}
	\frametitle{Correction 4}
\vspace*{1cm} 
 \footnotesize{Quelle est l'expression de la fonction affine passant par les points de coordonnées (8;-64) et (-4;32) ? Il existe deux techniques :} 
 \begin{multicols}{2} 
 \begin{enumerate} 
 \item On résout un système : $$ \begin{array}{rcl} 
 & \textcolor{white}{\Leftrightarrow} & 
 \left 
 \{\begin{array}{rcl}8\times m + p&=&-64 \\ 
 -4\times m+p&=&32\end{array} \right. \\ 
 &\Leftrightarrow & \left 
 \{\begin{array}{rcl} p&=&-64-8m \\ 
 -4m+p&=&32\end{array} \right. \\ 
 &\Leftrightarrow & \left 
 \{\begin{array}{rcl} p&=&-64-8m \\ 
 -4m+(-64-8m) &=&32\end{array} \right. \\ &\Leftrightarrow& \left \{\begin{array}{rcl}p&=&-64-8m \\ 
 -64-12m&=&32\end{array} \right. \\ &\Leftrightarrow& \left \{\begin{array}{rcl}p&=&-64-8m \\ 
 -12m&=&96\end{array} \right. \\  &\Leftrightarrow& \left \{\begin{array}{rcl} p&=&0 \\  m&=&-8\end{array}\right. \end{array}$$ 
 Ainsi on a $f:x\mapsto -8x$ 
 \columnbreak 
 \item 
 \footnotesize{On applique la formule du cours pour calculer $m$ :$$ \dfrac{f(x_1)-f(x_2)}{x_1-x_2}=\dfrac{-64-32}{8-\left(-4\right)}= \dfrac{-96}{12}=-8$$} \footnotesize{ Ainsi on a $f(x)= -8x +p $. 
  \\ On cherche maintenant la valeur de $p$. On sait que $f(8)=-64$. On doit donc résoudre $(E): -8\times8+p=-64$}	 
 \begin{align*} (E)& \Leftrightarrow -64+p=-64\\
		 	 & \Leftrightarrow p=-64+64\\
			 & \Leftrightarrow p=0
	 \end{align*} 
 Ainsi on a $f:x\mapsto -8x$ 
 \end{enumerate} 
 \end{multicols} 
 \end{frame}


\begin{frame}
\vspace{-10mm}
	\frametitle{Correction 5}
\vspace*{1cm} 
 \footnotesize{Quelle est l'expression de la fonction affine passant par les points de coordonnées (0;0) et (1;4) ? Il existe deux techniques :} 
 \begin{multicols}{2} 
 \begin{enumerate} 
 \item On résout un système : $$ \begin{array}{rcl} 
 & \textcolor{white}{\Leftrightarrow} & 
 \left 
 \{\begin{array}{rcl}0\times m + p&=&0 \\ 
 1\times m+p&=&4\end{array} \right. \\ 
 &\Leftrightarrow & \left 
 \{\begin{array}{rcl} p&=&00 \\ 
 m+p&=&4\end{array} \right. \\ 
 &\Leftrightarrow & \left 
 \{\begin{array}{rcl} p&=&00 \\ 
 m+(00 \\ 
  &=&4\end{array} \right. \\ &\Leftrightarrow& \left \{\begin{array}{rcl}p&=&00 \\ 
 0+m&=&4\end{array} \right. \\ &\Leftrightarrow& \left \{\begin{array}{rcl}p&=&00 \\ 
 m&=&4\end{array} \right. \\  &\Leftrightarrow& \left \{\begin{array}{rcl} p&=&0 \\  m&=&4\end{array}\right. \end{array}$$ 
 Ainsi on a $f:x\mapsto 4x$ 
 \columnbreak 
 \item 
 \footnotesize{On applique la formule du cours pour calculer $m$ :$$ \dfrac{f(x_1)-f(x_2)}{x_1-x_2}=\dfrac{0-4}{0-1}= \dfrac{-4}{-1}=4$$} \footnotesize{ Ainsi on a $f(x)= 4x +p $. 
  \\ On cherche maintenant la valeur de $p$. On sait que $f(0)=0$. On doit donc résoudre $(E): 4\times0+p=0$}	 
 \begin{align*} (E)& \Leftrightarrow 0+p=0\\
		 	 & \Leftrightarrow p=0\\
			 & \Leftrightarrow p=0
	 \end{align*} 
 Ainsi on a $f:x\mapsto 4x$ 
 \end{enumerate} 
 \end{multicols} 
 \end{frame}


\begin{frame}
\vspace{-10mm}
	\frametitle{Correction 6}
\[\left( 3^{-9}\right)^{10} =3^{\left(-9\right) \times 10} = 3^{-90}\]
\end{frame}




\end{document}